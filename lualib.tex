%%%%%%%%%%%%%%%%%%%%%%%%%%%%%%%%%%%%%%%%%%%%%%%%%%%%%%%%%%%%%%%
% Start of Luacode
%%%%%%%%%%%%%%%%%%%%%%%%%%%%%%%%%%%%%%%%%%%%%%%%%%%%%%%%%%%%%%%
\begin{luacode*}

function tikzDrawQRSVD(max_branch, origin_x, origin_y, radius, angle, lenght)
  --Left part
  tikztensorbranch(origin_x, origin_y-2.5, radius,-angle,lenght,'$n_{1}$','above right')
        tikztensorbranch(origin_x, origin_y-2.5, radius,-angle*2,lenght,'$n_{2}$','above right')
        tex.print(string.format("\\draw[->] (%f,%f) ++(7,-2.5) to node [above, sloped,font=\\footnotesize] {QR} ++(5,2.5);", origin_x, origin_y))
  
      --Right part
        origin_x = origin_x + 20
        tikztensorbranch(origin_x,origin_y,radius,-1*angle,lenght,'$n_{1}$','')
        tikztensorbranch(origin_x,origin_y,1,-2*angle,lenght,'','above right')
        --tikztensorbranch(origin_x,origin_y,1,180,0,'','')
        tikzlegend(origin_x+2, origin_y, 'above', '$r$')
        --tex.print(string.format("\\draw[fill=black] (%f,%f) circle (1cm);", origin_x,origin_y))
        --tex.print(string.format("\\draw[fill=white] (%f,%f+1) arc (90:270:1);", origin_x,origin_y))
        origin_x = origin_x + 4*radius
        tikztensorbranch(origin_x,origin_y,radius,-1*angle,lenght,'','above right')
        tikztensorbranch(origin_x,origin_y,radius,-2*angle,lenght,'$n_{2}$','right')
        tikztensorbranch(origin_x,origin_y,radius,400,0,'','below')

        local originX = 0;
        local originY = 15;
        local max_branch = 2;
        local angle = 360/max_branch;    
    --Left part
          tex.print(string.format("\\draw[->] (%f,%f) ++(7,+2.5) to node [below, sloped,font=\\footnotesize] {SVD} ++(5,-2.5);", originX, originY))
          local firstA = originX+18;
    
          --Right part
          tikztensorbranch(firstA,originY,1,-1*angle,2,'$n_{1}$','above right')
          tikztensorbranch(firstA,originY,1,-2*angle,2,'','above right')
          tikztensorbranch(firstA,originY,1,400,0,'','below')
          --tex.print("\\draw[fill=black] ("..firstA..","..originY..") circle (1cm);")
          --tex.print("\\draw[fill=white] ("..firstA..","..originY.."+1) arc (90:270:1);")
          tikzlegend(firstA+2, originY, 'above', '$r$')

          local firstB = firstA+4;
          tikztensorbranch(firstB,originY,1,-1*angle,2,'','above right')
          tikztensorbranch(firstB,originY,1,-2*angle,2,'','above right')
          tikztensorbranch(firstB,originY,1,400,0,'','below')
          tikzlegend(firstB+2, originY, 'above', '$r$')

          local firstC = firstB+4;
          tikztensorbranch(firstC,originY,1,-1*angle,2,'','above right')
          tikztensorbranch(firstC,originY,1,-2*angle,2,'$n_{2}$','above right')
          tikztensorbranch(firstC,originY,1,400,0,'','below')
          --tex.print("\\draw[fill=black] ("..firstC..","..originY.."+1) arc (90:270:1);")

end

function tikzDrawQR(origin_x, origin_y, radius, angle, lenght)
      --Left part
        tikztensorbranch(origin_x, origin_y, radius,-angle,lenght,'$n_{1}$','above right')
        tikztensorbranch(origin_x, origin_y, radius,-angle*2,lenght,'$n_{2}$','above right')
        tex.print(string.format("\\draw[->] (%f+5,%f) -- ++(5,0);", origin_x, origin_y))
        tex.print(string.format("\\node[] at (%f+6,%f+1.5) {QR};", origin_x, origin_y))
  
      --Right part
        origin_x = origin_x + 15
        tikztensorbranch(origin_x,origin_y,radius,-1*angle,lenght,'$n_{1}$','')
        tikztensorbranch(origin_x,origin_y,1,-2*angle,lenght,'','above right')
        --tikztensorbranch(origin_x,origin_y,1,180,0,'','')
        tikzlegend(origin_x+2, origin_y, 'above', '$r$')
        tex.print(string.format("\\draw[fill=black] (%f,%f) circle (1cm);", origin_x,origin_y))
        tex.print(string.format("\\draw[fill=white] (%f,%f+1) arc (90:270:1);", origin_x,origin_y))
        origin_x = origin_x + 4*radius
        tikztensorbranch(origin_x,origin_y,radius,-1*angle,lenght,'','above right')
        tikztensorbranch(origin_x,origin_y,radius,-2*angle,lenght,'$n_{2}$','right')
        tikztensorbranch(origin_x,origin_y,radius,400,0,'','below')
    end


function tikzDrawHOSVD(max_branch, originX, originY, angle, radius)
      for i = 1,max_branch,1 do
      local x = originX
      local lenght = 3
      local angle = -i*angle
      --Do the same line as above but withouth string.format
      if (i == max_branch) then
      local string = "\\draw[] ("..x.." ,"..originY.." ) arc (0:"..angle..":1.5) -- node[above,sloped] {\\tiny $r_{d}$} ++("..angle..":"..lenght..") --++("..angle..":1) coordinate (temp);"
      local string2 = "\\draw[fill=white] (temp) circle ("..radius.."-0.1) coordinate (t2);"
      local string3 = "\\draw[] (t2) ++("..angle..":"..radius..") -- ++("..angle..":1) ++("..angle..":1.5) node {$n_{d}$};"
      tex.print(string..string2..string3)
      elseif (i <= max_branch - 2) then
      local string = "\\draw[] ("..x.." ,"..originY.." ) arc (0:"..angle..":1.5) -- node[above,sloped] {\\tiny $r_{"..i.."}$} ++("..angle..":"..lenght..")  --++("..angle..":1) coordinate (temp);"
      local string2 = "\\draw[fill=white] (temp) circle ("..radius.."-0.1) coordinate (t2);"
      local string3 = "\\draw[] (t2) ++("..angle..":"..radius..") -- ++("..angle..":1) ++("..angle..":1.5) node {$n_{"..i.."}$};"
      tex.print(string..string2..string3)
      end
      target_branch = 1
      margin = 360/max_branch * 0.40
      fir_angle = 360/max_branch * (target_branch-1) + margin
      las_angle = 360/max_branch * (target_branch+1) - margin
      tikzDrawAlongCircle(4, originX, originY+1, 4, fir_angle, las_angle)
      end
    end

    function tikzSVD(max_branch, originX, originY, angle)
    --Left part
          for i = 1,max_branch,1 do
          tikztensorbranch(originX,originY,1,-i*angle,2,'$n_{'..i..'}$','above right')
          end
          tikztensorbranch(originX,originY,1,400,0,'','below')
          tex.print("\\draw[->] ("..originX.."+7,"..originY..") -- ("..originX.."+11,"..originY..");")
          tex.print("\\node at ("..originX.."+9,"..originY.."+1.5) {SVD};")
          local firstA = originX+18;
    
          --Right part
          tikztensorbranch(firstA,originY,1,-1*angle,2,'$n_{1}$','above right')
          tikztensorbranch(firstA,originY,1,-2*angle,2,'','above right')
          tikztensorbranch(firstA,originY,1,400,0,'','below')
          tex.print("\\draw[fill=black] ("..firstA..","..originY..") circle (1cm);")
          tex.print("\\draw[fill=white] ("..firstA..","..originY.."+1) arc (90:270:1);")
          tikzlegend(firstA+2, originY, 'above', '$r$')

          local firstB = firstA+4;
          tikztensorbranch(firstB,originY,1,-1*angle,2,'','above right')
          tikztensorbranch(firstB,originY,1,-2*angle,2,'','above right')
          tikztensorbranch(firstB,originY,1,400,0,'','below')
          tikzlegend(firstB+2, originY, 'above', '$r$')

          local firstC = firstB+4;
          tikztensorbranch(firstC,originY,1,-1*angle,2,'','above right')
          tikztensorbranch(firstC,originY,1,-2*angle,2,'$n_{2}$','above right')
          tikztensorbranch(firstC,originY,1,400,0,'','below')
          tex.print("\\draw[fill=black] ("..firstC..","..originY.."+1) arc (90:270:1);")
    end

function tikzDrawAlongCircle(nb_dots, origin_x, origin_y, radius, angle1, angle2)
    local function draw_dot_at_angle(angle)
      local x = origin_x + radius * math.cos(math.rad(angle))
      local y = origin_y + radius * math.sin(math.rad(angle))
      tex.sprint(string.format("\\filldraw[black] (%f,%f) circle (%f pt);", x, y, radius))
    end
    local angle_step = (angle2 - angle1) / (nb_dots-1)
    for i=0,nb_dots-1 do
      draw_dot_at_angle(angle1 + i * angle_step)
    end
  end
  
    function tikztensorbranch(origin_x, origin_y, radius, angle, lenght, label)
      local x = origin_x + radius
      local string = string.format("\\draw[] (%f ,%f ) arc (0:%f:1cm) -- ++(%f:%f) ++(%f:%f) node {%s};", 
      x, origin_y, angle, angle, lenght, angle, 1, label)
      tex.print(string)
    end
  
    function tikztensorbranchCustom(max_branch, origin_x, origin_y, radius, d_branch)
      local angle = 360/max_branch;
      for i = 1,max_branch,1 do
          if ( i == max_branch -1 and max_branch > 3 and d_branch) then
          elseif (i == max_branch and max_branch > 3 and d_branch)  then
            tikztensorbranch(origin_x,origin_y,radius,-i*angle,2,'$n_{d}$')
            target_branch = 1
            margin = 360/max_branch * 0.40
            fir_angle = 360/max_branch * (target_branch-1) + margin
            las_angle = 360/max_branch * (target_branch+1) - margin          
            tikzDrawAlongCircle(4, origin_x, origin_y, 3.5, fir_angle, las_angle)
          elseif (i == max_branch and d_branch)  then
            tikztensorbranch(origin_x,origin_y,radius,-i*angle,2,'$n_{d}$')
            target_branch = 1
            margin = 360/max_branch * 0.25
            fir_angle = 360/max_branch * (target_branch-1) + margin
            las_angle = 360/max_branch * (target_branch) - margin          
            tikzDrawAlongCircle(3, origin_x, origin_y, 3.5, fir_angle, las_angle)
          else
            tikztensorbranch(origin_x,origin_y,1,-i*angle,2,'$n_{'..i..'}$','above left')
          end
        end
        tikztensorbranch(origin_x,origin_y,1,400,0,'')
    end
  
    -- function tikzDrawQR(origin_x, origin_y, radius, angle, lenght)
    --   --Left part
    --     tikztensorbranch(origin_x, origin_y, radius,-angle,lenght,'$n_{1}$','above right')
    --     tikztensorbranch(origin_x, origin_y, radius,-angle*2,lenght,'$n_{2}$','above right')
    --     tex.print(string.format("\\draw[->] (%f+7,%f) -- ++(3,0);", origin_x, origin_y))
    --     tex.print(string.format("\\node[] at (%f+8,%f+1) {QR};", origin_x, origin_y))
  
    --   --Right part
    --     origin_x = origin_x + 18
    --     tikztensorbranch(origin_x,origin_y,radius,-1*angle,lenght,'$n_{1}$','')
    --     tikztensorbranch(origin_x,origin_y,1,-2*angle,lenght,'','above right')
    --     --tikztensorbranch(origin_x,origin_y,1,180,0,'','')
    --     tikzlegend(origin_x+2, origin_y, 'above', '$r$')
    --     tex.print(string.format("\\draw[fill=black] (%f,%f) circle (1cm);", origin_x,origin_y))
    --     tex.print(string.format("\\draw[fill=white] (%f,%f+1) arc (90:270:1);", origin_x,origin_y))
    --     origin_x = origin_x + 4*radius
    --     tikztensorbranch(origin_x,origin_y,radius,-1*angle,lenght,'','above right')
    --     tikztensorbranch(origin_x,origin_y,radius,-2*angle,lenght,'$n_{2}$','right')
    --     tikztensorbranch(origin_x,origin_y,radius,400,0,'','below')
    -- end
  
  
    function tikztensortrain(max_branch, origin_x, origin_y, radius, d_branch)
      local branch_lenght = 2*radius;
      local x = origin_x;
      local height_rank = 1;
      for i = 1,max_branch,1 do
        if (i == max_branch) then
          if (d_branch) then
            tex.print("\\draw[] ("..x..","..origin_y..") node {...};")
            tex.print("\\draw[] ("..x..","..origin_y+branch_lenght..") node {...};")
            x = x + radius
            local string3 = "\\draw[] ("..x..","..origin_y..") -- ++("..branch_lenght..",0);"
            local lbranch = "\\draw[] ("..x..","..origin_y..") ++("..branch_lenght.."/2,"..height_rank..") node {\\scriptsize $r_{d-1}$};"
            x = x + branch_lenght + radius
            local string = "\\draw[] ("..x..","..origin_y..") circle ("..radius..");"
            local string2 = "\\draw[] ("..x..","..origin_y+radius..") -- ++(0,"..branch_lenght..") ++(0,0.5) node {$n_{d}$};"
            tex.print(lbranch..string3..string..string2)
          else
            local string = "\\draw[] ("..x..","..origin_y..") circle ("..radius..");"
            local string2 = "\\draw[] ("..x..","..origin_y+radius..") -- ++(0,"..branch_lenght..") ++(0,0.5) node {$n_{"..i.."}$};"
            tex.print(string..string2)
          end
        else
          local string = "\\draw[] ("..x..","..origin_y..") circle ("..radius..");"
          local string2 = "\\draw[] ("..x..","..origin_y+radius..") -- ++(0,"..branch_lenght..") ++(0,0.5) node {$n_{"..i.."}$};"
          local string3 = "\\draw[] ("..x+radius..","..origin_y..") -- ++("..branch_lenght..",0);"
          local string4 = "\\draw[] ("..x+radius..","..origin_y..") ++("..branch_lenght.."/2,"..height_rank..") node {\\scriptsize $r_{"..i.."}$};"
          tex.print(string..string2..string3..string4)
          x = x + 2*branch_lenght;
        end
      end
    end

      function tikzTensorTrainInnerP(max_branch, origin_x, origin_y, radius, d_branch)
      local branch_lenght = 2*radius;
      local x = origin_x;
      local upperY = origin_y+2*branch_lenght;
      local height_rank = 1;
      for i = 1,max_branch,1 do
        if (i == max_branch) then
          if (d_branch) then
            tex.print("\\draw[] ("..x..","..upperY..") node {...};")
            tex.print("\\draw[] ("..x..","..origin_y..") node {...};")
            tex.print("\\draw[] ("..x..","..origin_y+branch_lenght..") node {...};")
            x = x + radius
            local string4 = "\\draw[] ("..x..","..upperY..") -- ++("..branch_lenght..",0);"
            local ubranch = "\\draw[] ("..x..","..upperY..") ++("..branch_lenght.."/2,"..height_rank..") node {\\scriptsize $q_{d-1}$};"
            local string3 = "\\draw[] ("..x..","..origin_y..") -- ++("..branch_lenght..",0);"
            local lbranch = "\\draw[] ("..x..","..origin_y..") ++("..branch_lenght.."/2,"..height_rank..") node {\\scriptsize $r_{d-1}$};"
            x = x + branch_lenght + radius
            local string1 = "\\draw[] ("..x..","..upperY..") circle ("..radius..");"
            local string = "\\draw[] ("..x..","..origin_y..") circle ("..radius..");"
            local string2 = "\\draw[] ("..x..","..origin_y+radius..") -- ++(0,"..branch_lenght..") ++(0,-"..branch_lenght.."/2) node [right] {$n_{d}$};"
            tex.print(lbranch..ubranch..string3..string..string2..string1..string4)
          else
            local string1 = "\\draw[] ("..x..","..upperY..") circle ("..radius..");"
            local string = "\\draw[] ("..x..","..origin_y..") circle ("..radius..");"
            local string2 = "\\draw[] ("..x..","..origin_y+radius..") -- ++(0,"..branch_lenght..") ++(0,-"..branch_lenght.."/2) node [right] {$n_{"..i.."}$};"
            tex.print(string1..string..string2)
          end
        else
          local string1 = "\\draw[] ("..x..","..upperY..") circle ("..radius..");"
          local string = "\\draw[] ("..x..","..origin_y..") circle ("..radius..");"
          local string2 = "\\draw[] ("..x..","..origin_y+radius..") -- ++(0,"..branch_lenght..") ++(0,-"..branch_lenght.."/2) node [right] {$n_{"..i.."}$};"
          local string5 = "\\draw[] ("..x+radius..","..upperY..") -- ++("..branch_lenght..",0);"
          local string3 = "\\draw[] ("..x+radius..","..origin_y..") -- ++("..branch_lenght..",0);"
          local string6 = "\\draw[] ("..x+radius..","..upperY..") ++("..branch_lenght.."/2,"..height_rank..") node {\\scriptsize $q_{"..i.."}$};"
          local string4 = "\\draw[] ("..x+radius..","..origin_y..") ++("..branch_lenght.."/2,"..height_rank..") node {\\scriptsize $r_{"..i.."}$};"
          tex.print(string1..string..string2..string5..string3..string6..string4)
          x = x + 2*branch_lenght;
        end
      end
    end

  function tikzlegend (originX, originY, pos, text)
  local legend = string.format ("\\node[%s] at (%s,%s) {%s};", pos, originX, originY, text)
  tex.print(legend)
  end

  function tikzcolor (red, green, blue)
  local color = string.format ("{rgb:red,%s;green,%s;blue,%s}", red, green, blue)
  return color
  end

  function tikzarrow (origin_x, origin_y, width, label)
  local arrow = string.format ("\\draw[-stealth](%s,%s) -- (%s+%s/2,%s) node[above]{%s} -- (%s+%s,%s);",
  origin_x, origin_y, origin_x, width, origin_y, label, origin_x, width, origin_y)
  tex.print(arrow)
  end

  function tikzarrow_down (origin_x, origin_y, width, style, label)
  local arrow = string.format ("\\draw [%s](%s,%s) -- (%s,%s-%s);",
  style, origin_x, origin_y, origin_x, origin_y, width, label, origin_x, origin_y, width)
  local name = string.format ("\\node[right] at (%s,%s-%s/2) {%s};",
  origin_x, origin_y, width, label)
  tex.print(arrow..name)
  end

  function tikzrank (origin_x, origin_y, height, label, style, color)
  local rank = string.format ("\\draw[%s, %s] (%s,%s) -- (%s,%s+%s);",
  style, color, origin_x, origin_y, origin_x, origin_y, height)
  local name = string.format ("\\node[%s] at (%s,%s+%s+0.2) {%s};", 
  color, origin_x, origin_y, height, label)
  tex.print(rank..name)
  end

  function tikzrank_horiz (origin_x, origin_y, height, label, style, color)
  local rank = string.format ("\\draw[%s, %s] (%s,%s) -- (%s+%s,%s);",
  style, color, origin_x, origin_y+0.2, origin_x, height, origin_y+0.2)
  local name = string.format ("\\node[%s] at (%s-0.2,%s) {%s};", 
  color, origin_x, origin_y+0.2, label)
  tex.print(rank..name)
  end

  function tikzpolygon(origin_x, origin_y, width, height, label, style, fill)
  if (fill == nil or fill == '') then
    fill = 'white'
  end
  local code1 = string.format("\\draw[style=%s,fill=%s](%s,%s)--(%s+%s,%s) -- (%s+%s,%s+%s) -- (%s,%s+%s) --cycle;",
  style, fill, origin_x, origin_y, origin_x, width, origin_y, origin_x, width,
  origin_y,height, origin_x, origin_y, height)
  local code2 =
  string.format("\\node[%s] at (%s+%s/2,%s+%s+0.5) {%s};", style, origin_x,width, origin_y, height, label)
  tex.print(code1..code2)
  end

  function tikzfactor(origin_x, origin_y, width, height, rank, label_U,
  label_V,
  style, fill)
  local factorU = tikzpolygon(origin_x, origin_y, width/2+rank, height,
  label_U,
  style, fill)
  local factorV = tikzpolygon(origin_x+(width/2+rank)+0.5, origin_y+2.2, width,
  height/2+rank, label_V,
  style, fill)
  return origin_x+(width/2+rank)+1
  end

  function tikzminus(origin_x, origin_y, width)
  local minus = string.format("\\draw [](%s,%s) --(%s+%s,%s);",
  origin_x, origin_y, origin_x, width, origin_y)
  tex.print(minus)
  end

  function tikzplus(origin_x, origin_y, width, height)
  local hori = string.format("\\draw [](%s,%s) --(%s+%s,%s);",
  origin_x, origin_y, origin_x, width, origin_y)
  local verti = string.format("\\draw [](%s+%s/2, %s-%s) --(%s+%s/2, %s+%s);",
  origin_x, width, origin_y, height, origin_x, width, origin_y, height)
  tex.print(hori..verti)
  end

  function tikzequal(origin_x, origin_y, width, height)
  local upper = string.format("\\draw [](%s,%s+%s) --(%s+%s,%s+%s);",
  origin_x, origin_y, height, origin_x, width, origin_y, height)
  local lower = string.format("\\draw [](%s,%s) --(%s+%s,%s);",
  origin_x, origin_y, origin_x, width, origin_y)
  tex.print(upper..lower)
  end
\end{luacode*}
